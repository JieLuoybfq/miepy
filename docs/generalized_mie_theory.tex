\documentclass[11pt]{article}

% standard header for all latex files
\input{/home/john/.latex/standard_header.tex}

\usepackage[margin=.9in]{geometry}
\usepackage[labelfont=bf]{caption}
\usepackage{subcaption}
\usepackage{indentfirst}
\usepackage[affil-it]{authblk}
\usepackage{enumitem}
\usepackage{scrextend}
\usepackage{cleveref}

\usepackage[nottoc,notlot,notlof]{tocbibind}
\renewcommand{\bibname}{References}

\title{Generalized multiparticle Mie theory}
\date{}
\author{}

\begin{document}
\maketitle

\section{Vector spherical harmonic functions}

The generalized multiparticle Mie theory (GMMT) is outlined, following Xu's work. \cite{xu1995electromagnetic}
The vector spherical harmonic (VSH) functions are a complete basis set of the vector wave equations
\begin{align}
    \begin{split}
        \nabla \times \nabla \times \boldsymbol{E} &= k^2 \boldsymbol{E} \\
        \nabla \times \nabla \times \boldsymbol{H} &= k^2 \boldsymbol{H}
    \end{split}
\end{align}
They are defined as
\begin{align}
    \boldsymbol{M}_{mn}^{(J)} &= \big[
    \boldsymbol{\hat \theta} i\pi_{mn}(\cos\theta) 
   -\boldsymbol{\hat \phi} \tau_{mn}(\cos\theta) \big] z_n^{(J)}(kr) e^{im\phi} \\
    \boldsymbol{N}_{mn}^{(J)} &= 
    \boldsymbol{\hat r} n(n+1) P_n^m(\cos\theta) \frac{z_n^{(J)}(kr)}{kr}e^{im\phi} \\
    &\quad + \frac{1}{kr} \left[ \boldsymbol{\hat \theta} \tau_{mn}(\cos\theta) + \boldsymbol{\hat \phi} i\pi_{mn}(\cos\theta) \right]
    \frac{d}{dr} \left[ rz_n^{(J)}(kr)e^{im\phi} \right] \nonumber
\end{align}
where $J= 1,2,3,4$. The radial functions $z_n^{(J)}$ are
\begin{equation}
\begin{align*}
    z_n^{(1)}(x) &= j_n(x)  \qquad&  z_n^{(3)}(x) &= h_n^{(1)}(x) = j_n(x) + iy_n(x)   \\
z_n^{(2)}(x) &= y_n(x)  \qquad&  z_n^{(4)}(x) &= h_n^{(2)}(x) = j_n(x) - iy_n(x)
\end{align*}
\end{equation}
where $j_n$, $y_n$ are the spherical Bessel functions of the first and second kind, and $h_n^{(1)}$, $h_n^{(2)}$ are the spherical Hankel functions of the first and second kind.
The angular functions $\pi_{mn}$ and $\tau_{mn}$ are
\begin{align}
\pi_{mn}(\cos\theta) &= \frac{m}{\sin\theta} P_n^m(\cos\theta) \\
\tau_{mn}(\cos\theta) &= \frac{d}{d\theta} P_n^m(\cos\theta) 
\end{align}
The associated Legendre polynomials $P_n^m$ are defined without the Condon-Shortley phase, i.e.
\begin{equation}
P_n^m(x) = (1 - x^2)^{m/2} \frac{d^m}{dx^m} P_n(x)
\end{equation}
The VSHs are an orthogonal set when integrated over a closed surface $\Omega$
\begin{align}
\begin{split}
    \langle \boldsymbol{N}_{mn}^{(J)}, \boldsymbol{N}_{m^\prime n^\prime}^{(J)} \rangle
    &= \int_\Omega \boldsymbol{N}_{mn}^{(J)} \cdot \boldsymbol{N}_{m^\prime n^\prime}^{(J)*} \;d\Omega \\
    &= \delta_{mm^\prime}\delta_{nn^\prime}4\pi \frac{n(n+1)(n+m)!}{(2n+1)(n-m)!}
      \left[ \frac{\left|z_n^{(J)}(kr) + krz_n^{(J)\prime}(kr)\right|^2 + n(n+1) \left|z_n^{(J)}(kr)\right|^2 }{(kr)^2} \right] \\
    \langle \boldsymbol{M}_{mn}^{(J)}, \boldsymbol{M}_{m^\prime n^\prime}^{(J)} \rangle
    &= \int_\Omega \boldsymbol{M}_{mn}^{(J)} \cdot \boldsymbol{M}_{m^\prime n^\prime}^{(J)*} \;d\Omega
    = \delta_{mm^\prime}\delta_{nn^\prime}4\pi \frac{n(n+1)(n+m)!}{(2n+1)(n-m)!} |z_n^{(J)}(kr)|^2 \\
    \langle \boldsymbol{N}_{mn}^{(J)}, \boldsymbol{M}_{m^\prime n^\prime}^{(J)} \rangle
    &= \int_\Omega \boldsymbol{N}_{mn}^{(J)} \cdot \boldsymbol{M}_{m^\prime n^\prime}^{(J)*} \;d\Omega = 0
\end{split}
\end{align}


\section{Field expansions}
\textit{Electric field}
\begin{subequations}
\begin{align}
    \boldsymbol{E}_\text{src}^j &= - \sum_{n=1}^{N_\text{max}} \sum_{m=-n}^{n}
    iE_{mn} \left[ p_{mn}^{j,\text{src}} \boldsymbol{N}_{mn}^{(1)} + q_{mn}^{j,\text{src}} \boldsymbol{M}_{mn}^{(1)} \right] \\
    \boldsymbol{E}_\text{inc}^j &= - \sum_{n=1}^{N_\text{max}} \sum_{m=-n}^{n}
    iE_{mn} \left[ p_{mn}^j \boldsymbol{N}_{mn}^{(1)} + q_{mn}^j \boldsymbol{M}_{mn}^{(1)} \right] \\
    \boldsymbol{E}_\text{scat}^j &= \sum_{n=1}^{N_\text{max}} \sum_{m=-n}^{n}
    iE_{mn} \left[a_{mn}^j \boldsymbol{N}_{mn}^{(3)} + b_{mn}^j \boldsymbol{M}_{mn}^{(3)} \right] \\
    \boldsymbol{E}_\text{int}^j &= - \sum_{n=1}^{N_\text{max}} \sum_{m=-n}^{n}
    iE_{mn} \left[ d_{mn}^j \boldsymbol{N}_{mn}^{(1)} + c_{mn}^j \boldsymbol{M}_{mn}^{(1)} \right]
\end{align}
\end{subequations}

\textit{Magnetic field}
\begin{subequations}
\begin{align}
    \boldsymbol{H}_\text{src}^j &= - \sqrt{\frac{\varepsilon_b}{\mu_b}} \sum_{n=1}^{N_\text{max}} \sum_{m=-n}^{n}
    E_{mn} \left[ q_{mn}^{j,\text{src}} \boldsymbol{N}_{mn}^{(1)} + p_{mn}^{j,\text{src}} \boldsymbol{M}_{mn}^{(1)} \right] \\
    \boldsymbol{H}_\text{inc}^j &= - \sqrt{\frac{\varepsilon_b}{\mu_b}} \sum_{n=1}^{N_\text{max}} \sum_{m=-n}^{n}
    E_{mn} \left[ q_{mn}^j \boldsymbol{N}_{mn}^{(1)} + p_{mn}^j \boldsymbol{M}_{mn}^{(1)} \right] \\
    \boldsymbol{H}_\text{scat}^j &= \sqrt{\frac{\varepsilon_b}{\mu_b}} \sum_{n=1}^{N_\text{max}} \sum_{m=-n}^{n}
    E_{mn} \left[b_{mn}^j \boldsymbol{N}_{mn}^{(3)} + a_{mn}^j \boldsymbol{M}_{mn}^{(3)} \right] \\
    \boldsymbol{H}_\text{int}^j &= -\sqrt{\frac{\varepsilon^j}{\mu^j}}  \sum_{n=1}^{N_\text{max}} \sum_{m=-n}^{n}
    E_{mn} \left[ c_{mn}^j \boldsymbol{N}_{mn}^{(1)} + d_{mn}^j \boldsymbol{M}_{mn}^{(1)} \right]
\end{align}
\end{subequations}
where $a_{mn}^j = a_n^j p_{mn}^j$, $b_{mn}^j = b_n^j q_{mn}^j$, $c_{mn}^j = c_n^j q_{mn}^j$, and $d_{mn}^j = d_n^j p_{mn}^j$; $a_n^j$, $b_n^j$, $c_n^j$, and $d_n^j$ are the ordinary Mie coefficients \cite{bohren2008absorption} of particle $j$, and $E_{mn}$ is a normalization factor
\begin{equation}
    E_{mn} = i^n \sqrt{\frac{(2n+1)(n-m)!}{n(n+1)(n+m)!}}
\end{equation}

\section{VSH translation addition coefficients}
\section{Interaction equations}
\begin{align}
    \begin{split}
        p_{mn}^j &= p_{mn}^{j,\text{src}}  -  \sum_{l \neq j}^{(1,N)}\sum_{v=1}^{N_\text{max}} \sum_{u=-v}^{v}
    A_{mn}^{uv}(l \rightarrow j) a_v^l p_{uv}^{l}
    +B_{mn}^{uv}(l \rightarrow j) b_v^l q_{uv}^{l} \\
    q_{mn}^j &= q_{mn}^{j,\text{src}}  -  \sum_{l \neq j}^{(1,N)}\sum_{v=1}^{N_\text{max}} \sum_{u=-v}^{v}
    B_{mn}^{uv}(l \rightarrow j) a_v^l p_{uv}^{l}
    +A_{mn}^{uv}(l \rightarrow j) b_v^l q_{uv}^{l}
    \label{eqn:gmt_system}
    \end{split}
\end{align}
where $A_{mn}^{uv}(l \rightarrow j)$ and $B_{mn}^{uv}(l \rightarrow j)$ are VSH translation coefficients from particle $l$ to particle $j$.

\section{Force and torque}
\emph{Maxwell stress tensor equations}
\begin{equation}
    \langle \boldsymbol{T} \rangle = \frac{1}{2} \text{Re} \left[ \varepsilon  \boldsymbol{E} \otimes \boldsymbol{E^*} + \mu \boldsymbol{H} \otimes \boldsymbol{H^*}
    - \frac{1}{2}(\varepsilon E^2 + \mu H^2)\boldsymbol{I} \right]
\end{equation}

\begin{equation}
    \langle \boldsymbol{F} \rangle = \oint_\Omega \langle \boldsymbol{T} \rangle \cdot d \boldsymbol{\Omega}
\end{equation}

\begin{equation}
    \langle \boldsymbol{\tau} \rangle = \oint_\Omega \boldsymbol{r} \times \langle \boldsymbol{T}  \rangle \cdot d \boldsymbol{\Omega}
\end{equation}

\emph{Force equations}
\begin{subequations}
\begin{align}
\begin{split}
    F_x + iF_y =& \frac{\pi}{k^2} \sum_{n=1}^{N_\text{max}} \sum_{m=-n}^{n} \frac{1}{n+1}\bigg\{
          \frac{\sqrt{(n+m+1)(n-m)}}{n}\frac{\varepsilon_b}{\mu_b}
          \bigg[2a_{mn}b_{m+1n}^*  - a_{mn}q_{m+1n}^* \\ 
        & - p_{mn} b_{m+1n}^* + 2b_{mn}a_{m+1n}^* - b_{mn}p_{m+1n}^* - q_{mn}a_{m+1n}^*  \bigg] \\
        & - \sqrt{\frac{(n+m+2)(n+m+1)n(n+2)}{(2n+3)(2n+1)}}
        \bigg[ 2 \varepsilon_b a_{mn}a_{m+1n+1}^* - \varepsilon_b a_{mn}p_{m+1n+1}^* \\
        & - \varepsilon_b p_{mn}a_{m+1n+1}^* + 2 \frac{\varepsilon_b}{\mu_b} b_{mn}b_{m+1n+1}^* - \frac{\varepsilon_b}{\mu_b} b_{mn}q_{m+1n+1}^* - \frac{\varepsilon_b}{\mu_b} q_{mn}b_{m+1n+1}^*\bigg] \\
        & + \sqrt{\frac{(n-m+1)(n-m+2)n(n+2)}{(2n+3)(2n+1)}}
        \bigg[ 2 \varepsilon_b a_{m-1n+1}a_{mn}^* - \varepsilon_b a_{m-1n+1}p_{mn}^* \\
        & - \varepsilon_b p_{m-1n+1}a_{mn}^* + 2 \frac{\varepsilon_b}{\mu_b} b_{m-1n+1}b_{mn}^* - \frac{\varepsilon_b}{\mu_b} b_{m-1n+1}q_{mn}^* - \frac{\varepsilon_b}{\mu_b} q_{m-1n+1}b_{mn}^*\bigg]
        \bigg\}
\end{split}
\end{align}

\begin{align}
\begin{split}
    F_z =& -\frac{2\pi}{k^2} \sum_{n=1}^{N_\text{max}} \sum_{m=-n}^{n} \frac{1}{n+1}\text{Re}\bigg\{
          \frac{m}{n} \frac{\varepsilon_b}{\mu_b}
          \bigg[ 2a_{mn}b_{mn}^* - a_{mn}q_{mn}^* - p_{mn}b_{mn}^* \bigg] \\
        & + \sqrt{\frac{(n-m+1)(n+m+1)n(n+2)}{(2n+3)(2n+1)}}
        \bigg[ 2 \varepsilon_b a_{mn+1}a_{mn}^* - \varepsilon_b a_{mn+1}p_{mn}^* - \varepsilon_b p_{mn+1}a_{mn}^* \\
        & + 2 \frac{\varepsilon_b}{\mu_b} b_{mn+1}b_{mn}^* - \frac{\varepsilon_b}{\mu_b} b_{mn+1}q_{mn}^* - \frac{\varepsilon_b}{\mu_b} q_{mn+1}b_{mn}^*
        \bigg] \bigg\}
\end{split}
\end{align}
\end{subequations}

\emph{Torque equations}
\begin{subequations}
\begin{align}
\begin{split}
    \tau_x =& \frac{2\pi}{k^3} \sum_{n=1}^{N_\text{max}} \sum_{m=-n}^{n} \sqrt{(n-m)(n+m+1)} \; \text{Re} \bigg\{
            \varepsilon_b a_{mn}a_{m+1n}^* + \mu_b b_{mn}b_{m+1n}^* \\
            & - \frac{1}{2} \bigg[ \varepsilon_b a_{m+1n}p_{mn}^* + \varepsilon_b a_{mn}p_{m+1n}^*
            + \mu_b b_{m+1n}q_{mn}^* + \mu_b b_{mn}q_{m+1n}^*\bigg] \bigg\}
\end{split}
\end{align}

\begin{align}
\begin{split}
    \tau_y =& \frac{2\pi}{k^3} \sum_{n=1}^{N_\text{max}} \sum_{m=-n}^{n} \sqrt{(n-m)(n+m+1)} \; \text{Im} \bigg\{
            \varepsilon_b a_{mn}a_{m+1n}^* + \mu_b b_{mn}b_{m+1n}^* \\
            & + \frac{1}{2} \bigg[ \varepsilon_b a_{m+1n}p_{mn}^* - \varepsilon_b a_{mn}p_{m+1n}^*
            + \mu_b b_{m+1n}q_{mn}^* - \mu_b b_{mn}q_{m+1n}^*\bigg] \bigg\}
\end{split}
\end{align}

\begin{align}
\begin{split}
    \tau_z =& -\frac{2\pi}{k^3} \sum_{n=1}^{N_\text{max}} \sum_{m=-n}^{n} m \bigg\{
          \varepsilon_b |a_{mn}|^2 + \mu_b |b_{mn}|^2 - \text{Re} \bigg[
              \varepsilon_b a_{mn}p_{mn}^* + \mu_b b_{mn}q_{mn}^*\bigg] \bigg\}
\end{split}
\end{align}
\end{subequations}

\section{Cluster coefficients}
\begin{align}
\begin{split}
    a_{mn} &= \sum_{l=1}^N\sum_{v=1}^{N_\text{max}} \sum_{u=-v}^{v}
    A_{mn}^{uv}(l \rightarrow \boldsymbol{p_0}) a_{uv}^{l}
    +B_{mn}^{uv}(l \rightarrow \boldsymbol{p_0}) b_{uv}^{l} \\
    b_{mn} &= \sum_{l=1}^N\sum_{v=1}^{N_\text{max}} \sum_{u=-v}^{v}
    B_{mn}^{uv}(l \rightarrow \boldsymbol{p_0}) a_{uv}^{l}
    +A_{mn}^{uv}(l \rightarrow \boldsymbol{p_0}) b_{uv}^{l}
\end{split}
\end{align}

\section{Far-field expansions}

\begin{align}
\begin{split}
    E_{\text{scat},\theta}(\theta,\phi) &= \frac{e^{ikr}}{-ikr} \sum_{n=1}^{N_\text{max}} \sum_{m=-n}^{n}
    E_{mn} \big[a_{mn}\tau_{mn}(\cos\theta) + b_{mn}\pi_{mn}(\cos\theta)\big] e^{im\phi} \\
    E_{\text{scat},\phi}(\theta,\phi) &= i\frac{e^{ikr}}{-ikr} \sum_{n=1}^{N_\text{max}} \sum_{m=-n}^{n}
    E_{mn} \big[a_{mn}\pi_{mn}(\cos\theta) + b_{mn}\tau_{mn}(\cos\theta)\big] e^{im\phi}
\end{split}
\end{align}

\section{Cross-sections}

The cross-sections can be computed via two methods: one that uses the cluster coefficients $a_{mn}$, $b_{mn}$ and one that uses the individual particle coefficients $a_{mn}^j$, $b_{mn}^j$.

\hfill

\textit{Cross-sections via individual particle coefficients} \cite{xu1997electromagnetic}
\begin{subequations}
\begin{align}
    C_\text{abs} &= \frac{4\pi}{k^2} \sum_{j=1}^N \sum_{n=1}^{N_\text{max}} \sum_{m=-n}^{n}
    D_n^j|a_{mn}^j|^2 + C_n^j|b_{mn}^j|^2 \\
    C_\text{ext} &= \frac{4\pi}{k^2} \sum_{j=1}^N \sum_{n=1}^{N_\text{max}} \sum_{m=-n}^{n}
    \text{Re} \bigg\{ p_{mn}^{j,\text{src}*} a_{mn}^j 
    + q_{mn}^{j,\text{src}*}b_{mn}^j \bigg\} \\
    C_\text{scat} &= C_\text{ext} - C_\text{abs}
\end{align}
\end{subequations}
where
\begin{subequations}
\begin{align}
    D_n^j &= \frac{\text{Re}\{ im^j \mu_b \mu^j \psi_n(y^j)\psi_n^{\prime *}(y^j)\}}
    {|\mu_b m^j \psi_n(y^j)\psi_n^\prime(x^j) - \mu^j \psi_n(x^j)\psi_n^\prime(y^j)|^2} \\
    C_n^j &= \frac{\text{Re}\{ im^{j*} \mu_b \mu^j \psi_n(y^j)\psi_n^{\prime *}(y^j)\}}
    {|\mu^j \psi_n(y^j)\psi_n^\prime(x^j) - \mu_b m^j \psi_n(x^j)\psi_n^\prime(y^j)|^2}
\end{align}
\end{subequations}
and $m_j = n^j/n_b$, $x^j = k r^j$, $y^j = m^jx^j$ and $\psi_n$ is the Riccati-Bessel function of the first kind.
This approach is the most efficient way to compute the total cross-sections.
Each term in the absorption/extinction cross-section sum can be interpreted as the absorption/extinction of that individual particle due to a given mode $(n,m)$

\hfill

\textit{Cross-sections via cluster coefficients} \cite{xu1995electromagnetic}
\begin{subequations}
\begin{align}
    C_\text{scat} &= \frac{4\pi}{k^2} \sum_{n=1}^{N_\text{max}} \sum_{m=-n}^{n}
    |a_{mn}|^2 + |b_{mn}|^2 \\
    C_\text{ext} &= \frac{4\pi}{k^2} \sum_{n=1}^{N_\text{max}} \sum_{m=-n}^{n}
    \text{Re} \bigg\{ p_{mn}^{\text{src}*}(\boldsymbol{p_0}) a_{mn} 
    + q_{mn}^{\text{src}*}(\boldsymbol{p_0})b_{mn} \bigg\} \\
    C_\text{abs} &= C_\text{ext} - C_\text{scat}
\end{align}
\end{subequations}
This approach has a benefit in its interpretation.
Each term in the scattering sum corresponds to the multipolar scattering of order $(n,m)$, so that the scattering from the entire cluster can be identified as electric or magnetic in nature, dipole, quadrupole, etc.
These equations should typically be avoided in calculating total cross-sections since there is a loss of information in using the cluster coefficients and they may not converge.

All of these cross-sections have units of (area)$\times$(electric field)$^2$.
If the source is a plane wave of amplitude $E_0$, then these cross-sections should be normalized by $E_0^2$.
For non-plane wave sources, the cross-sections should be normalized depending on the convention being used, typically an averaged intensity over some area:
\begin{equation}
    E_0^2 = \frac{1}{A} \int_A |\boldsymbol{E}(\boldsymbol{r})|^2 \;dA
\end{equation}

\section{Source decomposition}
Given the fields of the incident source $\boldsymbol{E}_\text{src}(\boldsymbol{r})$, $\boldsymbol{H}_\text{src}(\boldsymbol{r})$, the source can be decomposed by integration
\begin{align}
\begin{split}
    p_{mn}^{j,\text{src}} &= i\cfrac{\int_\Omega \boldsymbol{E}_\text{src} \cdot \boldsymbol{N}_{mn}^{(1)*} \; d\Omega}
    {E_{mn} \langle \boldsymbol{N}_{mn}^{(1)},\boldsymbol{N}_{mn}^{(1)} \rangle} \\
    q_{mn}^{j,\text{src}} &= i\cfrac{\int_\Omega \boldsymbol{E}_\text{src} \cdot \boldsymbol{M}_{mn}^{(1)*} \; d\Omega}
    {E_{mn} \langle \boldsymbol{M}_{mn}^{(1)},\boldsymbol{M}_{mn}^{(1)} \rangle}
\end{split}
\end{align}
where $\Omega$ is a closed surface around particle $j$.

\section{Efficient numeric evaluation}
\section{Recursion relations}

\section{Other conventions}
A different convention for the field expansions used in other work \cite{barton1989theoretical} is presented here.
These field expansions were used to evaluate analytic expressions for the force and torque.
\newline

\textit{Electric field}
\begin{subequations}
\begin{align}
\begin{split}
    \boldsymbol{E}_\text{inc}^j = \sum_{n=1}^{N_\text{max}} \sum_{m=-n}^{n} \bigg\{
    \boldsymbol{\hat r}\frac{1}{r^2} &\bigg[ n(n+1) p_{mn}^j \psi_n(kr) Y_{nm}(\theta,\phi) \bigg] \\
    +\; \boldsymbol{\hat \theta}\frac{k}{r} &\left[ p_{mn}^j \psi_n^\prime(kr) \frac{\partial}{\partial \theta} Y_{nm}(\theta,\phi)
    - \frac{m}{\sqrt{\varepsilon_b}} q_{mn}^j \psi_n(kr) \frac{Y_{nm}(\theta,\phi)}{\sin\theta} \right] \\
    +\; \boldsymbol{\hat \phi}\frac{k}{r} &\left[ im p_{mn}^j \psi_n^\prime(kr) \frac{Y_{nm}(\theta,\phi)}{\sin\theta}
    - \frac{i}{\sqrt{\varepsilon_b}} q_{mn}^j \psi_n(kr) \frac{\partial}{\partial \theta} Y_{nm}(\theta,\phi) \right] \bigg\}
\end{split}
\end{align}
\begin{align}
\begin{split}
    \boldsymbol{E}_\text{scat}^j = \sum_{n=1}^{N_\text{max}} \sum_{m=-n}^{n} \bigg\{
    \boldsymbol{\hat r}\frac{1}{r^2} &\bigg[ n(n+1) a_{mn}^j \xi_n^{(1)}(kr) Y_{nm}(\theta,\phi) \bigg] \\
    +\; \boldsymbol{\hat \theta}\frac{k}{r} &\left[ a_{mn}^j \xi_n^{(1)\prime}(kr) \frac{\partial}{\partial \theta} Y_{nm}(\theta,\phi)
    - \frac{m}{\sqrt{\varepsilon_b}} b_{mn}^j \xi_n^{(1)}(kr) \frac{Y_{nm}(\theta,\phi)}{\sin\theta} \right] \\
    +\; \boldsymbol{\hat \phi}\frac{k}{r} &\left[ im a_{mn}^j \xi_n^{(1)\prime}(kr) \frac{Y_{nm}(\theta,\phi)}{\sin\theta}
    - \frac{i}{\sqrt{\varepsilon_b}} b_{mn}^j \xi_n^{(1)}(kr) \frac{\partial}{\partial \theta} Y_{nm}(\theta,\phi) \right] \bigg\}
\end{split}
\end{align}
\end{subequations}

\textit{Magnetic field}
\begin{subequations}
\begin{align}
\begin{split}
    \boldsymbol{H}_\text{inc}^j = \sum_{n=1}^{N_\text{max}} \sum_{m=-n}^{n} \bigg\{
    \boldsymbol{\hat r}\frac{1}{r^2} &\bigg[ n(n+1) q_{mn}^j \psi_n(kr) Y_{nm}(\theta,\phi) \bigg] \\
    +\; \boldsymbol{\hat \theta}\frac{k}{r} &\left[ q_{mn}^j \psi_n^\prime(kr) \frac{\partial}{\partial \theta} Y_{nm}(\theta,\phi)
    + m\sqrt{\varepsilon_b} p_{mn}^j \psi_n(kr) \frac{Y_{nm}(\theta,\phi)}{\sin\theta} \right] \\
    +\; \boldsymbol{\hat \phi}\frac{k}{r} &\left[ im q_{mn}^j \psi_n^\prime(kr) \frac{Y_{nm}(\theta,\phi)}{\sin\theta}
    + i\sqrt{\varepsilon_b} p_{mn}^j \psi_n(kr) \frac{\partial}{\partial \theta} Y_{nm}(\theta,\phi) \right] \bigg\}
\end{split}
\end{align}
\begin{align}
\begin{split}
    \boldsymbol{H}_\text{scat}^j = \sum_{n=1}^{N_\text{max}} \sum_{m=-n}^{n} \bigg\{
    \boldsymbol{\hat r}\frac{1}{r^2} &\bigg[ n(n+1) b_{mn}^j \xi_n^{(1)}(kr) Y_{nm}(\theta,\phi) \bigg] \\
    +\; \boldsymbol{\hat \theta}\frac{k}{r} &\left[ n_{mn}^j \xi^{(1)\prime}(kr) \frac{\partial}{\partial \theta} Y_{nm}(\theta,\phi)
    + m\sqrt{\varepsilon_b} a_{mn}^j \xi_n^{(1)}(kr) \frac{Y_{nm}(\theta,\phi)}{\sin\theta} \right] \\
    +\; \boldsymbol{\hat \phi}\frac{k}{r} &\left[ im b_{mn}^j \xi_n^{(1)\prime}(kr) \frac{Y_{nm}(\theta,\phi)}{\sin\theta}
    + i\sqrt{\varepsilon_b} a_{mn}^j \xi_n^{(1)}(kr) \frac{\partial}{\partial \theta} Y_{nm}(\theta,\phi) \right] \bigg\}
\end{split}
\end{align}
\end{subequations}
where $\xi_n^{(1)} = \psi_n - i \chi_n$, $\psi_n$, $\chi_n$ are the Riccati-Bessel function of the first and second kind, and $Y_{nm}$ are the spherical harmonics
\begin{align}
\begin{split}
    \psi_n(x) &= xj_n(x) \\
    \chi_n(x) &= -xy_n(x) \\
    \xi_n^{(1)}(x) &= x[j_n(x) + iy_n(x)] = xh_n^{(1)}(x) \\
    Y_{nm}(\theta, \phi) &= \sqrt{\frac{2n+1}{4\pi}\frac{(n-m)!}{(n+m)!}} P_n^m(\cos \theta) e^{im\phi}
\end{split}
\end{align}

Denoting the coefficients of our convention $\bar a_n$, $\bar b_n$, $\bar p_{mn}$, $\bar q_{mn}$, $\bar a_{mn}$, $\bar b_{mn}$, the two conventions are related by
\begin{align}
\begin{split}
    \bar a_{n} &= - a_{n} \\
    \bar b_{n} &= - b_{n} \\
    \bar p_{mn} &= \frac{k^2}{i^{n-1}}\sqrt{\frac{n(n+1)}{4\pi}} p_{mn} \\
    \bar q_{mn} &= -\frac{k^2}{i^n}\sqrt{\frac{\mu_b}{\varepsilon_b} \frac{n(n+1)}{4\pi}} q_{mn} \\
    \bar a_{mn} &= -\frac{k^2}{i^{n-1}}\sqrt{\frac{n(n+1)}{4\pi}} a_{mn} \\
    \bar b_{mn} &= \frac{k^2}{i^n}\sqrt{\frac{\mu_b}{\varepsilon_b} \frac{n(n+1)}{4\pi}} b_{mn}
\end{split}
\end{align}
Another convention uses a different value for the $E_{mn}$ normalization values \cite{xu1995electromagnetic}
\begin{equation}
    E_{mn} = |E_0|i^n \frac{2n+1}{n(n+1)}
\end{equation}
where $|E_0|$ is the amplitude of the source field.
We have chosen to absorb this amplitude into the $a$, $b$, $p$, and $q$ coefficients.

\section{MiePy}

Example Python code using MiePy
\begin{lstlisting}
    import miepy
    nm = 1e-9

    # define material and source
    Ag = miepy.materials.predefined.Ag()
    source = miepy.sources.x_polarized_plane_wave()

    # build an Ag dimer with radii 50nm separated by 600nm in the x-direction
    spheres = miepy.spheres([[300*nm,0,0], [-300*nm,0,0]], radius=50*nm, material=Ag)
    cluster = miepy.gmt(spheres, source, wavelength=800*nm, Lmax=2)

    # obtain the cross-sections
    scat, absorb, extinct = cluster.cross_sections()

    # obtain the force and torque on the right particle
    F = cluster.force_on_particle(0)
    T = cluster.torque_on_particle(0)
\end{lstlisting}


\bibliographystyle{unsrt}
\bibliography{generalized_mie_theory.bib}{}

\end{document}
