\documentclass[11pt]{article}

\usepackage{amsmath,amsfonts,amssymb}   % basic math
\usepackage{graphicx}  % images
\usepackage{float}     % floating image support
\usepackage{listings}  % code input
\usepackage{color}     % colors
\usepackage[super,comma]{natbib}    % citing
\usepackage[T1]{fontenc}      % fonts
\usepackage[utf8]{inputenc}   % fonts
\usepackage{titling}   % title control
\usepackage{microtype}
\usepackage{bm}

\usepackage[utf8]{inputenc}
\usepackage[margin=.9in]{geometry}
\usepackage[labelfont=bf]{caption}
\usepackage{subcaption}
\usepackage{indentfirst}
\usepackage[affil-it]{authblk}
\usepackage{enumitem}
\usepackage{scrextend}

\usepackage{varioref}
\usepackage{hyperref}
\usepackage{cleveref}
\hypersetup{
    colorlinks,
    citecolor=blue,
    filecolor=black,
    linkcolor=blue,
    urlcolor=blue
}

\bibliographystyle{unsrtnat}
\renewcommand{\bibname}{References}

% standard code colors
\definecolor{dkgreen}{rgb}{0,0.6,0}
\definecolor{gray}{rgb}{0.5,0.5,0.5}
\definecolor{mauve}{rgb}{0.58,0,0.82}

% setup for python language
\lstset{frame=tb,
  language=Python,
  aboveskip=3mm,
  belowskip=3mm,
  showstringspaces=false,
  columns=flexible,
  basicstyle={\small\ttfamily},
  numbers=none,
  numberstyle=\tiny\color{gray},
  keywordstyle=\color{blue},
  commentstyle=\color{gray},
  stringstyle=\color{dkgreen},
  breaklines=true,
  breakatwhitespace=true,
  tabsize=3
}

\title{Generalized multiparticle Mie theory (GMMT)}
\date{}
\author{}

\begin{document}
\maketitle
\tableofcontents

\section{Particle interactions in the GMMT}
\subsection{Vector spherical harmonic functions}

The generalized multiparticle Mie theory (GMMT) is outlined, following Xu's work. \cite{xu1995electromagnetic}
The vector spherical harmonic (VSH) functions are a complete basis set of the vector wave equations
\begin{align}
    \begin{split}
        \nabla \times \nabla \times \boldsymbol{E} &= k^2 \boldsymbol{E} \\
        \nabla \times \nabla \times \boldsymbol{H} &= k^2 \boldsymbol{H}
    \end{split}
\end{align}
They are defined as
\begin{align}
    \begin{split}
        \boldsymbol{N}_{mn1}^{(J)} &= 
        \boldsymbol{\hat r} n(n+1) P_n^m(\theta) \frac{z_n^{(J)}(kr)}{kr}e^{im\phi} \\
        &\quad + \frac{1}{kr} \left[ \boldsymbol{\hat \theta} \tau_{mn}(\theta) + \boldsymbol{\hat \phi} i\pi_{mn}(\theta) \right]
        \frac{d}{dr} \left[ rz_n^{(J)}(kr)e^{im\phi} \right] \\
        \boldsymbol{N}_{mn2}^{(J)} &= \big[
        \boldsymbol{\hat \theta} i\pi_{mn}(\theta) 
       -\boldsymbol{\hat \phi} \tau_{mn}(\theta) \big] z_n^{(J)}(kr) e^{im\phi} 
    \end{split}
\label{eqn:vsh_definition}
\end{align}
where $J= 1,2,3,4$. The radial functions $z_n^{(J)}$ are
\begin{align}
\begin{split}
    \begin{aligned}
        z_n^{(1)}(x) &= j_n(x)  \qquad&  z_n^{(3)}(x) &= h_n^{(1)}(x) = j_n(x) + iy_n(x)
    \end{aligned}\\
    \begin{aligned}
        z_n^{(2)}(x) &= y_n(x)  \qquad&  z_n^{(4)}(x) &= h_n^{(2)}(x) = j_n(x) - iy_n(x)
    \end{aligned}
\end{split}
\end{align}
where $j_n$, $y_n$ are the spherical Bessel functions of the first and second kind, and $h_n^{(1)}$, $h_n^{(2)}$ are the spherical Hankel functions of the first and second kind.
The angular functions $\pi_{mn}$ and $\tau_{mn}$ are
\begin{align}
\pi_{mn}(\theta) &= \frac{m}{\sin\theta} P_n^m(\cos\theta) \\
\tau_{mn}(\theta) &= \frac{d}{d\theta} P_n^m(\cos\theta) 
\end{align}
The associated Legendre polynomials $P_n^m$ are defined without the Condon-Shortley phase, i.e.\
\begin{equation}
P_n^m(x) = (1 - x^2)^{m/2} \frac{d^m}{dx^m} P_n(x)
\end{equation}

The VSHs are an orthogonal (but not orthonormal) set when integrated over a closed surface $\Omega$
\begin{align}
\begin{split}
    \langle \boldsymbol{N}_{mn1}^{(J)}, \boldsymbol{N}_{m^\prime n^\prime 1}^{(J)} \rangle
    &= \int_\Omega \boldsymbol{N}_{mn1}^{(J)} \cdot \boldsymbol{N}_{m^\prime n^\prime 1}^{(J)*} \;d\Omega \\
    &= \delta_{mm^\prime}\delta_{nn^\prime}4\pi \frac{n(n+1)(n+m)!}{(2n+1)(n-m)!}
      \left[ \frac{\left|z_n^{(J)}(kr) + krz_n^{(J)\prime}(kr)\right|^2 + n(n+1) \left|z_n^{(J)}(kr)\right|^2 }{(kr)^2} \right] \\
    \langle \boldsymbol{N}_{mn2}^{(J)}, \boldsymbol{N}_{m^\prime n^\prime 2}^{(J)} \rangle
    &= \int_\Omega \boldsymbol{N}_{mn2}^{(J)} \cdot \boldsymbol{N}_{m^\prime n^\prime 2}^{(J)*} \;d\Omega
    = \delta_{mm^\prime}\delta_{nn^\prime}4\pi \frac{n(n+1)(n+m)!}{(2n+1)(n-m)!} |z_n^{(J)}(kr)|^2 \\
    \langle \boldsymbol{N}_{mn1}^{(J)}, \boldsymbol{N}_{m^\prime n^\prime 2}^{(J)} \rangle
    &= \int_\Omega \boldsymbol{N}_{mn1}^{(J)} \cdot \boldsymbol{N}_{m^\prime n^\prime 2}^{(J)*} \;d\Omega = 0
\end{split}
\end{align}

The table below outlines the physical interpretation of the VSHs
\begin{center}
 \begin{tabular}{|l l|} 
 \hline
 \textbf{Entity} & \textbf{Physical interpretation}  \\ [0.5ex] 
 \hline\hline
 $\bm{N}_{mn1}$ & electric (TE) modes \\ 
 $\bm{N}_{mn2}$ & magnetic (TM) modes \\ 
 \hline
 $n$ & multipolar order (1: dipole, 2: quadrupole, etc.) \\
 $m$ & azimuthal order (from $-n$ to $n$) \\
 \hline
 $J=1$ & propagating incident mode \\ 
 $J=2$ & counter-propagating incident mode \\ 
 $J=3$ & spherically outgoing mode \\ 
 $J=4$ & spherically ingoing mode \\
 \hline
\end{tabular}
\end{center}

\subsection{Field expansions}
The source, incident, scattered, and interior electric and magnetic fields of particle $j$ can be expanded in terms of the VSHs.
The incident field includes the source field plus the incident field from all other particles in the system.
\newline

\textit{Electric field}
\begin{subequations}
\begin{align}
    \boldsymbol{E}_\text{src}^j &= - \sum_{n=1}^{N_\text{max}} \sum_{m=-n}^{n} \sum_{r=1}^2
    iE_{mn} p_{mnr}^{j,\text{src}} \boldsymbol{N}_{mnr}^{(1)} \\
    \boldsymbol{E}_\text{inc}^j &= - \sum_{n=1}^{N_\text{max}} \sum_{m=-n}^{n} \sum_{r=1}^2
    iE_{mn} p_{mnr}^{j,\text{inc}} \boldsymbol{N}_{mnr}^{(1)} \\
    \boldsymbol{E}_\text{scat}^j &= \sum_{n=1}^{N_\text{max}} \sum_{m=-n}^{n} \sum_{r=1}^2
    iE_{mn} p_{mnr}^{j,\text{scat}} \boldsymbol{N}_{mnr}^{(3)} \\
    \boldsymbol{E}_\text{int}^j &= - \sum_{n=1}^{N_\text{max}} \sum_{m=-n}^{n} \sum_{r=1}^2
    iE_{mn} p_{mnr}^{j,\text{int}} \boldsymbol{N}_{mnr}^{(1)}
\end{align}
\label{eqn:electric_field_expansion}
\end{subequations}

\textit{Magnetic field}
\begin{subequations}
\begin{align}
    \boldsymbol{H}_\text{src}^j &= - \sqrt{\frac{\varepsilon_b}{\mu_b}} \sum_{n=1}^{N_\text{max}} \sum_{m=-n}^{n} \sum_{r=1}^2
    E_{mn} p_{mn\bar r}^{j,\text{src}} \boldsymbol{N}_{mnr}^{(1)} \\
    \boldsymbol{H}_\text{inc}^j &= - \sqrt{\frac{\varepsilon_b}{\mu_b}} \sum_{n=1}^{N_\text{max}} \sum_{m=-n}^{n} \sum_{r=1}^2
    E_{mn} p_{mn\bar r}^{j,\text{inc}} \boldsymbol{N}_{mnr}^{(1)} \\
    \boldsymbol{H}_\text{scat}^j &= \sqrt{\frac{\varepsilon_b}{\mu_b}} \sum_{n=1}^{N_\text{max}} \sum_{m=-n}^{n} \sum_{r=1}^2
    E_{mn} p_{mn\bar r}^{j,\text{scat}} \boldsymbol{N}_{mnr}^{(3)} \\
    \boldsymbol{H}_\text{int}^j &= - \sqrt{\frac{\varepsilon^j}{\mu^j}} \sum_{n=1}^{N_\text{max}} \sum_{m=-n}^{n} \sum_{r=1}^2
    E_{mn} p_{mn\bar r}^{j,\text{int}} \boldsymbol{N}_{mnr}^{(1)}
\end{align}
\label{eqn:magnetic_field_expansion}
\end{subequations}
where $\bar r = 3-r$ and $E_{mn}$ is a normalization factor
\begin{equation}
    E_{mn} = i^n \sqrt{\frac{(2n+1)(n-m)!}{n(n+1)(n+m)!}}
\end{equation}

\subsection{VSH translation coefficients}
The VSH functions computed in one coordinate system ($\bm{r}_l$) can be converted to a different coordinate system ($\bm{r}_j$) by use of the translation coefficient $\widetilde{A}_{mnruvs}^{jl}$
\begin{equation}
    \bm{N}_{mnr}^{(J)}(k\bm{r}_j) = \sum_{v=1}^\infty \sum_{u=-v}^{u=v} \sum_{s=1}^2
    \widetilde{A}_{mnruvs}^{(J)jl} \bm{N}_{uvs}^{(1)}(k\bm{r}_l)
    \label{eqn:VSHW_translation}
\end{equation}
Here, the fields being translated from are incident fields (hence $J=1$), and the fields being translated to can correspond to any $J$ value.
If $J=1$, $\widetilde{A}$ translates incident fields to scattered fields, and if $J=3$, $\widetilde{A}$ translates incident fields to incident fields.
Explicit formula for the translation coefficients can be found elsewhere. \cite{Xu_1998}

From the field expansions, \cref{eqn:electric_field_expansion}, the VSH translation coefficients can be used to relate the incident expansion coefficients of particle $l$ to the expansion coefficients around particle $j$,
\begin{equation}
    p_{mnr}^{j,(J)} = f^{(J)}\sum_{v=1}^\infty \sum_{u=-v}^{u=v} \sum_{s=1}^2
    A_{mnruvs}^{(J)jl} p_{uvs}^{l,\text{inc}}
    \label{eqn:VSHW_translation_normalized}
\end{equation}
where $A_{mnruvs}^{(J)jl}$ are the \emph{normalized} translation coefficients
\begin{equation}
    A_{mnruvs}^{(J)jl} = \frac{E_{uv}}{E_{mn}} \widetilde{A}_{mnruvs}^{(J)jl}
\end{equation}
and $f^{(J)}$ is a $\pm 1$ sign term
\begin{equation}
    f^{(J)} =
    \begin{cases}
        +1 \qquad \text{if } J = 3 \\
        -1 \qquad \text{otherwise}
    \end{cases}
    \label{eqn:outgoing_factor}
\end{equation}
The translation coefficients have the following useful symmetry relationships \cite{hovenier1996light}
\begin{subequations}
    \label{eqn:vsh_symmetry}
    \begin{align}
        A_{mnruvs}^{ij} &= (-1)^{n+v+r+s} A_{mnruvs}^{ji} \\
                        &= (-1)^{n+v+m+u} A_{-uvs-mnr}^{ji} \\
                        &= (-1)^{m+u}[A_{uvsmnr}^{ji}]^* \label{eqn:vsh_symmetry_special} \footnotemark
    \end{align}
\end{subequations}
\footnotetext{
The conjugate in \cref{eqn:vsh_symmetry_special} applies to everything but the radial function $z_n^{(J)}$ appearing in the sum found in the definition of $A^{ij}$.
If $J=1$ or $J=2$ and the medium is non-absorbing, this exception can be ignored since $z_n$ is real.
}

\subsection{VSH rotation coefficients}
The rotation of the expansion coefficients is given by the Wigner-D matrix
\begin{equation}
    p_{mnr}^\prime = D_{ms}^{n}(\hat q) p_{mns}
\end{equation}
where $\hat q$ represents the rotation.

\subsection{T-matrix formulation}
\newcommand{\tmatrix}{\mathcal{T}}
The T-matrix of particle $j$, $\tmatrix_{mnruvs}^j$, relates the incident expansion coefficients to the scattered expansion coefficients for an arbitrary, non-spherical particle.
To simplify the index notation, a multi-index is used to denote a given mode, i.e.\ $\alpha = (m,n,r)$.
Greek letters are used to represent a multi-index.
Furthermore, Einstein notation is used so that repeated multi-indices are always summed over.
Then the T-matrix is defined as
\begin{equation}
    p_\alpha^{j,\text{scat}} = \tmatrix_{\alpha\beta}^j p_\beta^{j,\text{inc}}
\label{eqn:tmatrix_defintion}
\end{equation}
If the particle is a sphere, then the T-matrix is diagonal,
\begin{equation}
    \tmatrix_{\alpha\beta}^j = t_{rn}^j \delta_{\alpha\beta}
\label{eqn:tmatrix_sphere}
\end{equation}
where $t_{1n}^j = a_n^j$ and $t_{2n}^j = b_n^j$ are the classical Mie theory coefficients of the $j$th sphere.\cite{bohren2008absorption}

Once a particle's T-matrix is known, it does not have to be recomputed for a new orientation of the particle, provided the geometrical and material properties of the particle remain the same.
Given a particle T-matrix in one coordinate system, the T-matrix in a rotated coordinate system is
\begin{equation}
    \tmatrix_{m^\prime nr u^\prime vs}^\prime = D_{m^\prime m}^n(\hat q) \lbrack D_{u^\prime u}^v(\hat q) \rbrack^* \tmatrix_{mnruvs}
\end{equation}

A particle also has an \emph{internal} T-matrix, $\tmatrix^\text{int}$, that relates the internal expansion coefficients to the incident coefficients
\begin{equation}
\begin{equation}
    p_\alpha^{j,\text{int}} = \tmatrix_{\alpha\beta}^{j,\text{int}} p_\beta^{j,\text{inc}}
\label{eqn:tmatrix_internal_defintion}
\end{equation}


\subsection{Interaction equations}
The interaction equation among $N$ particles can be written as
\begin{equation}
    p_{\alpha}^{j,\text{inc}} = 
    p_{\alpha}^{j,\text{src}} -
    A_{\alpha \beta}^{(3)jl}
    \tmatrix_{\beta \gamma}^l
    p_{\gamma}^{l,\text{inc}}
\label{eqn:particle_interactions}
\end{equation}
More explicitly, dropping the Einstein summation rule:
\begin{equation}
    p_{\alpha}^{j,\text{inc}} = 
    p_{\alpha}^{j,\text{src}} -
    \sum_{l \neq j}^{(1,N)}
    \sum_{\beta} \sum_{\gamma}
    A_{\alpha \beta}^{(3)jl}
    \tmatrix_{\beta \gamma}^l
    p_{\gamma}^{l,\text{inc}}
\end{equation}
Even more explicitly, the multi-index can be unraveled:
\begin{equation}
    p_{mnr}^{j,\text{inc}} = 
    p_{mnr}^{j,\text{src}} -
    \sum_{l \neq j}^{(1,N)}\sum_{v=1}^{L_\text{max}} \sum_{u=-v}^{v}
    \sum_{s=1}^{2}
    \sum_{v^\prime=1}^{L_\text{max}} \sum_{u^\prime=-v^\prime}^{v^\prime}
    \sum_{s^\prime=1}^{2}
    A_{mnruvs}^{(3)jl}
    \tmatrix_{uvsu^\prime v^\prime s^\prime}^l
    p_{u^\prime v^\prime s^\prime}^{l,\text{inc}}
\end{equation}
If the system consists entirely of spheres, then using \cref{eqn:tmatrix_sphere}, the interaction equation becomes
\begin{equation}
    p_{mnr}^{j,\text{inc}} = p_{mn}^{j,\text{src}}  -  \sum_{l \neq j}^{(1,N)}\sum_{v=1}^{L_\text{max}} \sum_{u=-v}^{v} \sum_{s=1}^2
    A_{mnruvs}^{(3)jl} t_{sv}^l p_{uvs}^{l,\text{inc}}
    \label{eqn:sphere_interations}
\end{equation}

Whether the cluster is composed solely of spheres or more generally non-spherical particles, the interaction equation is always of the form
\begin{equation}
    p_{\alpha}^{j,\text{inc}} = 
    p_{\alpha}^{j,\text{src}} -
    \tmatrix_{\alpha \beta}^{jl}
    p_{\beta}^{l,\text{inc}}
\end{equation}
where $\tmatrix_{\alpha \beta}^{jl}$ is the \emph{particle aggregate T-matrix}.
By setting $p_\alpha^{j,\text{inc}} = \delta_{\alpha\beta} \delta_{jl} p_\beta^{l,\text{inc}}$, the interaction equation can be rewritten in the standard form for linear systems, $[A]x=b$,
\begin{equation}
    \left[ \delta_{\alpha\beta} \delta_{jl}
    + \tmatrix_{\alpha \beta}^{jl} \right]
    p_{\beta}^{l,\text{inc}} =
    p_{\alpha}^{j,\text{src}}
\end{equation}

\subsection{Interactions with a substrate}
The interactions of the particles with the interface of a substrate can be included in the GMMT.
This is done by simultaneously matching the boundary conditions on the surface of the particles (using VSHW functions) and on the interface (using a plane wave expansion). \cite{mackowski2008exact}
We introduce a reflection matrix $\widetilde{R}_{\alpha\beta}^{jl}$ from particle $l$ to $j$, similar to that of direct translations in \cref{eqn:VSHW_translation}
\begin{equation}
    \bm{N}_{\alpha}^{(3),\text{ref}}(k\bm{r}_j) = 
    \widetilde{R}_{\alpha\beta}^{jl} \bm{N}_{\beta}^{(1)}(k\bm{r}_l)
    \label{eqn:VSHW_reflection_translation}
\end{equation}
This equation relates the scattered, reflected VSHW functions in terms of incident VSHW functions around a different coordinate system.
Expressions for the $\widetilde{R}_{\alpha\beta}^{jl}$ matrix in terms of substrate material and particle location is provided elsewhere. \cite{mackowski2008exact}

As before, these reflection coefficients are normalized to relate to the expansion coefficients
\begin{equation}
    p_{\alpha}^{j,\text{ref,scat}} = f^{(J)}\sum_{v=1}^\infty \sum_{u=-v}^{u=v} \sum_{s=1}^2
    R_{\alpha\beta}^{(J)jl} p_{\beta}^{l,\text{inc}}
    \label{eqn:VSHW_reflection_translation_normalized}
\end{equation}
The interaction equations then become
\begin{equation}
    p_{\alpha}^{j,\text{inc}}= 
    p_{\alpha}^{j,\text{src}} + p_{\alpha}^{j,\text{src,ref}}  -
    \left( A_{\alpha \beta}^{(3)jl} + R_{\alpha \beta}^{jl} \right)
    \tmatrix_{\beta \gamma}^l
    p_{\gamma}^{l,\text{inc}}
\label{eqn:particle_interactions}
\end{equation}
where $p_{\alpha}^{j,\text{src,ref}}$ are the expansion coefficients of the source reflected off of the substrate.

This approach can be extended to multiple substrates (layered media) \cite{egel2014dipole}.
It can also be extended to non-spherical particles very close to the substrate where the Rayleigh hypothesis fails. \cite{egel2016light}
\section{Symmetries}
Symmetries in the system can be used to reduce the total number of interaction equations.
It is important to remember that not only do the particles have to satisfy a given symmetry (their position, size, orientation, and material), but that the incident source field must also satisfy it.
In many cases, a phase factor must be included with a given symmetry depending on the incident field.
For example, if the incident field is $x$-polarized and the mirror is in the $yz$ plane, the mirrored fields will be $-x$-polarized.
This can be corrected by including a $\pi$ phase factor in the mirror symmetry (multiplying by $-1$).
On the other hand, if the incident field is right-hand-circularly polarized and a mirror symmetry is imposed, the mirrored fields will be left-hand-circularly polarized.
This cannot be corrected by using a phase factor.

\subsection{Translational symmetry (periodic boundary conditions)}
If the unit cell consists of a single particle,
\begin{equation}
    p_{\alpha}^{\text{inc}} = 
    p_{\alpha}^{\text{src}} -
    \Phi^l
    A_{\alpha \beta}^{(3)0l}
    \tmatrix_{\beta \gamma}
    p_{\gamma}^{\text{inc}}
\end{equation}
where $l$ enumerates the unit cells and $\Phi^l$ is a phase factor.
For a plane wave with a given $\bm{k}$ vector, $\Phi^l = \exp(-i\bm{k} \cdot \bm{r}^l)$.
If the unit cell consists of $N$ particles, let $l$ enumerate the positions of the unit cell and $j$ enumerate the positions of the particles relative to the unit cell position
\begin{equation}
    p_{\alpha}^{j,\text{inc}} = 
    p_{\alpha}^{j,\text{src}} -
    \Phi^l
    A_{\alpha \beta}^{(3)j 0}
    A_{\beta \gamma}^{(3)0l}
    A_{\gamma \delta}^{(3)l j^\prime}
    \tmatrix_{\delta \epsilon}^{j^\prime}
    p_{\epsilon}^{j^\prime,\text{inc}}
\end{equation}
Since three VSHW translations are performed (one to the center of the unit cell, one to the origin, and one back to a particle), this approach comes with a loss of information.
This loss of information can be avoided by performing the translation directly, at the cost performance
\begin{equation}
    p_{\alpha}^{j,\text{inc}} = 
    p_{\alpha}^{j,\text{src}} -
    \Phi^{j^\prime}
    A_{\alpha \beta}^{(3)j j^\prime}
    \tmatrix_{\beta \gamma}^{j^\prime}
    p_{\gamma}^{j^\prime,\text{inc}}
\end{equation}

\subsection{Mirror symmetry}
Suppose that there is a mirror plane that passes through the origin and has a normal vector of $\bm{\hat x}$, $\bm{\hat y}$, or $\bm{\hat z}$.
Each particle at position $\bm{r}^j$ then has a corresponding mirror particle at position $\bm{r}^{\prime j}$.
By symmetry, the expansion coefficients of the mirror particle are related to those of the original particle
\begin{equation}
    p^{\prime j,\text{inc}}_{mnr} = \Phi (-1)^{r+1} p^{j,\text{inc}}_{-mnr}
    \label{eqn:mirror_symmetry}
\end{equation}
i.e.\ the mirror negates the azimuthal index $m$ and negates the TM modes ($r=2$).

\subsection{Rotational symmetry}
Suppose the system has a discrete rotational symmetry of order $R$.
If a particle exists at position $r^{j1}$, then the symmetric particles are at position $r^{jr}$, where $r = 2..R$.
By symmetry,
\begin{equation}
    p^{jr,\text{inc}}_{mnr} = D_{mun}(\theta_r) p^{j1,\text{inc}}_{mur}
    \label{eqn:rotational_symmetry}
\end{equation}
Note that this requires the polarization of the light to be circular.

\section{Cluster properties}

\subsection{Cluster coefficients and cluster T-matrix}
The expansion coefficients of the entire cluster, $p_{\alpha}^{\text{cluster}}$, are computed by translating the individual particle coefficients $p_{\alpha}^{j,\text{scat}}$ to the origin $\bm{r_0}$

\begin{equation}
    p_{\alpha}^{\text{cluster}} = 
    A_{\alpha \beta}^{(1)0j}
    p_{\beta}^{j,\text{scat}}
\label{eqn:cluster_coefficients}
\end{equation}
Note that $J=1$ in the translation coefficient since the translation is from the scattered fields around one origin to the scattered fields around a different origin.

The \emph{particle aggregate T-matrix} is defined by 
\begin{equation}
    \tmatrix_{\alpha\beta}^{jl} =
    A_{\alpha \gamma}^{(3)jl}
    \tmatrix_{\gamma \beta}^l
\label{eqn:particle_aggregate_tmatrix}
\end{equation}
and the interaction equation becomes
\begin{equation}
    p_{\alpha}^{j,\text{inc}} = 
    p_{\alpha}^{j,\text{src}} -
    \tmatrix_{\alpha \beta}^{jl}
    p_{\beta}^{l,\text{inc}}
\end{equation}

The \emph{cluster T-matrix} is defined such that
\begin{equation}
    p_\alpha^{\text{cluster}} = \tmatrix_{\alpha\beta}^{\text{cluster}} p_\beta^{0,\text{src}}
\label{eqn:cluster_tmatrix}
\end{equation}
That is, the cluster T-matrix treats the system of particles as if it were a single scattering object, hiding the internal details.

\subsection{Cross-sections}

The cross-sections can be computed via two methods: one that uses the cluster coefficients $p_\alpha^\text{cluster}$ and one that uses the individual particle coefficients $p_\alpha^j$.

\hfill

\textit{Cross-sections via individual particle coefficients}
\begin{subequations}
\begin{align}
    C_\text{abs} &= \frac{4\pi}{k^2} \sum_{j=1}^N \sum_{\alpha}
    \text{Re} \left\{(p_\alpha^{j,\text{inc}})^* p_\alpha^{j,\text{scat}} \right\} 
        - |p_\alpha^{j,\text{scat}}|^2\\
    C_\text{ext} &= \frac{4\pi}{k^2} \sum_{j=1}^N \sum_{\alpha}
    \text{Re} \left\{(p_\alpha^{j,\text{src}})^* p_\alpha^{j,\text{scat}} \right\} \\
    C_\text{scat} &= C_\text{ext} - C_\text{abs}
\end{align}
\label{eqn:cross_sections}
\end{subequations}
This approach is the most efficient way to compute the total cross-sections.
Each term in the absorption/extinction cross-section sum can be interpreted as the absorption/extinction of that individual particle due to a given mode $\alpha$.

\hfill

\textit{Cross-sections via cluster coefficients} \cite{xu1995electromagnetic}
\begin{subequations}
\begin{align}
    C_\text{scat} &= \frac{4\pi}{k^2} \sum_{\alpha}
        |p_{\alpha}^\text{cluster}|^2 \\
    C_\text{ext} &= \frac{4\pi}{k^2} \sum_{\alpha}
    \text{Re} \left\{ (p_\alpha^{0,\text{src}})^* p_\alpha^\text{cluster} \right\} \\
    C_\text{abs} &= C_\text{ext} - C_\text{scat}
\end{align}
\end{subequations}
This approach has a benefit in its interpretation.
Each term in the scattering sum corresponds to the multipolar scattering of the  $\alpha$ mode, so that the scattering from the entire cluster can be identified as electric or magnetic in nature, dipole, quadrupole, etc.
These equations should typically be avoided in calculating total cross-sections since there is a loss of information in using the cluster coefficients and they may not converge.

All of these cross-sections have units of (area)$\times$(electric field)$^2$.
If the source is a plane wave of amplitude $E_0$, then these cross-sections should be normalized by $E_0^2$.
For non-plane wave sources, the cross-sections should be normalized depending on the convention being used, typically an averaged intensity over some area:
\begin{equation}
    E_0^2 = \frac{1}{A} \int_A \boldsymbol{S}(\boldsymbol{r}) 
    \cdot \bm{\hat n} \;dA
\end{equation}
where $\bm{S} = \frac{1}{2} \bm{E} \times \bm{H^*}$ is the Poynting vector.
For instance, the area $A$ may be a circular aperture (power striking a structure) or all of space (total power of the source, provided it is finite).

\subsection{Force and torque}
\newcommand{\mst}{\langle \boldsymbol{T} \rangle}

The time-average force $\langle \bm{F} \rangle$ and torque $\langle \bm{\tau} \rangle$ on a particle can be determined by integrating the Maxwell stress tensor $\mst$ over a closed surface surrounding the particle: \cite{mst}
\begin{subequations}
\begin{align}
    \mst &= \frac{1}{2} \text{Re} \left[ \varepsilon_b  \boldsymbol{E} \otimes \boldsymbol{E^*} + \mu_b \boldsymbol{H} \otimes \boldsymbol{H^*}
    - \frac{1}{2}(\varepsilon_b E^2 + \mu_b H^2)\boldsymbol{I} \right] \\
    \langle \boldsymbol{F} \rangle &= \oint_\Omega \mst \cdot d \boldsymbol{\Omega} \\
    \langle \boldsymbol{\tau} \rangle &= \oint_\Omega \boldsymbol{r} \times \mst \cdot d \boldsymbol{\Omega}
\end{align}
\end{subequations}
where $\bm{I}$ is a 3x3 identitiy matrix and $\otimes$ is the vector outer product.

The electric and magnetic fields around particle $j$ can be calculated using the field expansions, \cref{eqn:electric_field_expansion,eqn:magnetic_field_expansion}.
The integration for the time-averaged force can then be carried out analytically in the far-field, resulting in a sum over the particle expansion coefficients. \cite{barton1989theoretical}
For these equations, it is helpful to denote
$a_{mn} = p_{mn1}^{j,\text{inc}}$,
$b_{mn} = p_{mn2}^{j,\text{inc}}$,
$p_{mn} = p_{mn1}^{j,\text{scat}}$, and
$q_{mn} = p_{mn2}^{j,\text{scat}}$. 
Then, the force is given by
\begin{subequations}
\begin{align}
\begin{split}
    F_x + iF_y =& \frac{\pi}{k^2} \sum_{n=1}^{N_\text{max}} \sum_{m=-n}^{n} \frac{1}{n+1}\bigg\{
          \frac{\sqrt{(n+m+1)(n-m)}}{n}\frac{\varepsilon_b}{\mu_b}
          \bigg[2a_{mn}b_{m+1n}^*  - a_{mn}q_{m+1n}^* \\ 
        & - p_{mn} b_{m+1n}^* + 2b_{mn}a_{m+1n}^* - b_{mn}p_{m+1n}^* - q_{mn}a_{m+1n}^*  \bigg] \\
        & - \sqrt{\frac{(n+m+2)(n+m+1)n(n+2)}{(2n+3)(2n+1)}}
        \bigg[ 2 \varepsilon_b a_{mn}a_{m+1n+1}^* - \varepsilon_b a_{mn}p_{m+1n+1}^* \\
        & - \varepsilon_b p_{mn}a_{m+1n+1}^* + 2 \frac{\varepsilon_b}{\mu_b} b_{mn}b_{m+1n+1}^* - \frac{\varepsilon_b}{\mu_b} b_{mn}q_{m+1n+1}^* - \frac{\varepsilon_b}{\mu_b} q_{mn}b_{m+1n+1}^*\bigg] \\
        & + \sqrt{\frac{(n-m+1)(n-m+2)n(n+2)}{(2n+3)(2n+1)}}
        \bigg[ 2 \varepsilon_b a_{m-1n+1}a_{mn}^* - \varepsilon_b a_{m-1n+1}p_{mn}^* \\
        & - \varepsilon_b p_{m-1n+1}a_{mn}^* + 2 \frac{\varepsilon_b}{\mu_b} b_{m-1n+1}b_{mn}^* - \frac{\varepsilon_b}{\mu_b} b_{m-1n+1}q_{mn}^* - \frac{\varepsilon_b}{\mu_b} q_{m-1n+1}b_{mn}^*\bigg]
        \bigg\}
\end{split}
\end{align}

\begin{align}
\begin{split}
    F_z =& -\frac{2\pi}{k^2} \sum_{n=1}^{N_\text{max}} \sum_{m=-n}^{n} \frac{1}{n+1}\text{Re}\bigg\{
          \frac{m}{n} \frac{\varepsilon_b}{\mu_b}
          \bigg[ 2a_{mn}b_{mn}^* - a_{mn}q_{mn}^* - p_{mn}b_{mn}^* \bigg] \\
        & + \sqrt{\frac{(n-m+1)(n+m+1)n(n+2)}{(2n+3)(2n+1)}}
        \bigg[ 2 \varepsilon_b a_{mn+1}a_{mn}^* - \varepsilon_b a_{mn+1}p_{mn}^* - \varepsilon_b p_{mn+1}a_{mn}^* \\
        & + 2 \frac{\varepsilon_b}{\mu_b} b_{mn+1}b_{mn}^* - \frac{\varepsilon_b}{\mu_b} b_{mn+1}q_{mn}^* - \frac{\varepsilon_b}{\mu_b} q_{mn+1}b_{mn}^*
        \bigg] \bigg\}
\end{split}
\end{align}
\end{subequations}
Similarly, the torque can be integrated analytically,
\begin{subequations}
\begin{align}
\begin{split}
    \tau_x =& \frac{2\pi}{k^3} \sum_{n=1}^{N_\text{max}} \sum_{m=-n}^{n} \sqrt{(n-m)(n+m+1)} \; \text{Re} \bigg\{
            \varepsilon_b a_{mn}a_{m+1n}^* + \mu_b b_{mn}b_{m+1n}^* \\
            & - \frac{1}{2} \bigg[ \varepsilon_b a_{m+1n}p_{mn}^* + \varepsilon_b a_{mn}p_{m+1n}^*
            + \mu_b b_{m+1n}q_{mn}^* + \mu_b b_{mn}q_{m+1n}^*\bigg] \bigg\}
\end{split}
\end{align}

\begin{align}
\begin{split}
    \tau_y =& \frac{2\pi}{k^3} \sum_{n=1}^{N_\text{max}} \sum_{m=-n}^{n} \sqrt{(n-m)(n+m+1)} \; \text{Im} \bigg\{
            \varepsilon_b a_{mn}a_{m+1n}^* + \mu_b b_{mn}b_{m+1n}^* \\
            & + \frac{1}{2} \bigg[ \varepsilon_b a_{m+1n}p_{mn}^* - \varepsilon_b a_{mn}p_{m+1n}^*
            + \mu_b b_{m+1n}q_{mn}^* - \mu_b b_{mn}q_{m+1n}^*\bigg] \bigg\}
\end{split}
\end{align}

\begin{align}
\begin{split}
    \tau_z =& -\frac{2\pi}{k^3} \sum_{n=1}^{N_\text{max}} \sum_{m=-n}^{n} m \bigg\{
          \varepsilon_b |a_{mn}|^2 + \mu_b |b_{mn}|^2 - \text{Re} \bigg[
              \varepsilon_b a_{mn}p_{mn}^* + \mu_b b_{mn}q_{mn}^*\bigg] \bigg\}
\end{split}
\end{align}
\end{subequations}

\subsection{Far-field expansions}

In the far-field, the VSH functions take on a simplified form from their exact expression in \cref{eqn:vsh_definition}.
The radial component of $\bm{N}$ vanishes and the spherical Bessel functions approach an asymptotic form.
For spherically outgoing modes, the spherical Hankel function of the first kind is $h_n^{(1)}(kr) \simeq i^{-n-1} e^{ikr}/(kr)$ as $kr \rightarrow \infty$.

Given a set of scattering expansion coefficients $q_{mnr}$, the far fields can be expanded over a single sum of the multipolar modes:
\begin{align}
\begin{split}
    E_{\text{scat},\theta}(\theta,\phi) &= i\frac{e^{ikr}}{kr} \sum_{n=1}^{N_\text{max}} \sum_{m=-n}^{n}
    (-i)^nE_{mn} \big[q_{mn1}\tau_{mn}(\cos\theta) + q_{mn2}\pi_{mn}(\cos\theta)\big] e^{im\phi} \\
    E_{\text{scat},\phi}(\theta,\phi) &= -\frac{e^{ikr}}{kr} \sum_{n=1}^{N_\text{max}} \sum_{m=-n}^{n}
    (-i)^nE_{mn} \big[q_{mn1}\pi_{mn}(\cos\theta) + q_{mn2}\tau_{mn}(\cos\theta)\big] e^{im\phi} \\
\end{split}
\end{align}
One option is to set $q_{mnr} = p_{mnr}^\text{cluster}$, the cluster expansion coefficients.
This approach comes with a loss of information, but has the physical interpretation that each term in the sum can be attributed to the far-fields of a given multipolar mode.
To avoid the loss of information, a sum over particle pairs can be performed \cite{Xu_1997}
\begin{equation}
    q_{\alpha}(\theta, \phi) = \sum_{l=1}^N \sum_{j=1}^N \sum_\beta 
                 \exp \left[i(\bm{\hat k} - \bm{\hat r}) \cdot \bm{r}^l\right]
                 \exp \left[i \bm{\hat k} \cdot (\bm{r}^l - \bm{r}^j) \right]
                 \tmatrix_{\alpha \beta}^{lj} p_{\beta}^\text{j,\text{inc}}
\end{equation}
where the expansion coefficients now depend on the angular position.

\subsection{Orientation-averaged properties}

\section{Source decomposition}
To solve the particle interaction equation, \cref{eqn:particle_interactions}, the expansion coefficients of the source field must be known at each particle.
Given the electric fields of the incident source, $\boldsymbol{E}^\text{src}(\boldsymbol{r})$, the source can be decomposed into expansion coefficients by integration around particle $j$
\begin{equation}
    p_{mnr}^{j,\text{src}} = i\cfrac{\int_\Omega \boldsymbol{E}^\text{src} \cdot \boldsymbol{N}_{mnr}^{(1)*} \; d\Omega}
    {E_{mn} \langle \boldsymbol{N}_{mnr}^{(1)},\boldsymbol{N}_{mnr}^{(1)} \rangle} \\
\end{equation}
where $\Omega$ is a closed surface around particle $j$.

For a plane wave, this decomposition can be carried out analytically.
The direction of the incident $\bm{k}$ vector can be described by the two spherical angles $\theta$ and $\phi$.
A linear polarization of the plane wave is either TE ($\bm{\hat \theta}$) or TM ($\bm{\hat \phi}$).
Elliptically polarized light is then a complex-valued linear combination of TE and TM polarizations.
\begin{align}
\begin{split}
p_{mn}^{j,\text{src}} &= E_0 i^{-n} E_{mn} \tau_{mn}(\cos \alpha) \exp(-im\beta) \exp(i\bm{k} \cdot \bm{r}_j) \\
q_{mn}^{j,\text{src}} &= E_0 i^{-n} E_{mn} \pi_{mn}(\cos \alpha) \exp(-im\beta) \exp(i\bm{k} \cdot \bm{r}_j)
\end{split}
\end{align}


\subsection{Far-field integration}
\subsection{Near-field point matching}
\subsection{Focused beams}

\subsection{Reflection off of an interface}

\section{Efficient numerical implementation}
\subsection{T-matrix evaluation}
The T-matrix can be numerically evaluated using the so-called extended boundary condition method (EBCM). \cite{waterman1965matrix, barber1975scattering, mishchenko1996t}
In this method, the expansion coefficients are related to one another by the equivalence principle
\begin{align}
\begin{split}
    p^\text{inc}_\alpha &= iQ^{31}_{\alpha\beta} p^\text{int}_\beta \\
    p^\text{scat}_\alpha &= -iQ^{11}_{\alpha\beta} p^\text{int}_\beta
\end{split}
\end{align}
where the $Q$ matrices are given by integrals of the VSHW functions over the particle's surface
\begin{equation}
    Q_{\alpha \beta}^{pq} = -f^{(p)}\frac{ik_m^2}{\pi} 
    \left[ \int_S (d\bm{S} \times \bm{N}_{\beta}^{(q)}(k\bm{r})) \cdot \bm{N}_{\hat\alpha}^{(p)}(k_m \bm{r})
       + \sqrt{\frac{\varepsilon}{\varepsilon_m}}\int_S (d\bm{S} \times \bm{N}_{\hat\beta}^{(q)}(k\bm{r})) \cdot \bm{N}_{\alpha}^{(p)}(k_m \bm{r}) \right]
\end{equation}
where $f^{(p)}$ is defined in \cref{eqn:outgoing_factor} and the index notation $\hat \alpha$ means to swap electric and magnetic mode indices, i.e. if $\alpha = (m, n, r)$, then $\hat \alpha = (m, n, 3-r)$.
The T-matrices are then determined
\begin{align}
    \tmatrix_{\alpha\beta} &= -Q^{11}_{\alpha\gamma} \left[ Q^{31}_{\gamma\beta} \right]^{-1} \\
    \tmatrix_{\alpha\beta}^\text{int} &= -i\left[ Q^{31}_{\alpha\beta} \right]^{-1}
\end{align}
\subsection{Evaluation of VSH transition coefficients}
\subsection{VSH rotation-translation-rotation algorithm}
\subsection{Matrix-free iterative solver}
\subsection{Preconditioner based on spatial locality}
\subsection{Parallel execution}

\section{Other conventions}
A different convention for the field expansions used in other work \cite{barton1989theoretical} is presented here.
These field expansions were used to evaluate analytic expressions for the force and torque.
\newline

\textit{Electric field}
\begin{subequations}
\begin{align}
\begin{split}
    \boldsymbol{E}_\text{inc}^j = \sum_{n=1}^{N_\text{max}} \sum_{m=-n}^{n} \bigg\{
    \boldsymbol{\hat r}\frac{1}{r^2} &\bigg[ n(n+1) p_{mn}^j \psi_n(kr) Y_{nm}(\theta,\phi) \bigg] \\
    +\; \boldsymbol{\hat \theta}\frac{k}{r} &\left[ p_{mn}^j \psi_n^\prime(kr) \frac{\partial}{\partial \theta} Y_{nm}(\theta,\phi)
    - \frac{m}{\sqrt{\varepsilon_b}} q_{mn}^j \psi_n(kr) \frac{Y_{nm}(\theta,\phi)}{\sin\theta} \right] \\
    +\; \boldsymbol{\hat \phi}\frac{k}{r} &\left[ im p_{mn}^j \psi_n^\prime(kr) \frac{Y_{nm}(\theta,\phi)}{\sin\theta}
    - \frac{i}{\sqrt{\varepsilon_b}} q_{mn}^j \psi_n(kr) \frac{\partial}{\partial \theta} Y_{nm}(\theta,\phi) \right] \bigg\}
\end{split}
\end{align}
\begin{align}
\begin{split}
    \boldsymbol{E}_\text{scat}^j = \sum_{n=1}^{N_\text{max}} \sum_{m=-n}^{n} \bigg\{
    \boldsymbol{\hat r}\frac{1}{r^2} &\bigg[ n(n+1) a_{mn}^j \xi_n^{(1)}(kr) Y_{nm}(\theta,\phi) \bigg] \\
    +\; \boldsymbol{\hat \theta}\frac{k}{r} &\left[ a_{mn}^j \xi_n^{(1)\prime}(kr) \frac{\partial}{\partial \theta} Y_{nm}(\theta,\phi)
    - \frac{m}{\sqrt{\varepsilon_b}} b_{mn}^j \xi_n^{(1)}(kr) \frac{Y_{nm}(\theta,\phi)}{\sin\theta} \right] \\
    +\; \boldsymbol{\hat \phi}\frac{k}{r} &\left[ im a_{mn}^j \xi_n^{(1)\prime}(kr) \frac{Y_{nm}(\theta,\phi)}{\sin\theta}
    - \frac{i}{\sqrt{\varepsilon_b}} b_{mn}^j \xi_n^{(1)}(kr) \frac{\partial}{\partial \theta} Y_{nm}(\theta,\phi) \right] \bigg\}
\end{split}
\end{align}
\end{subequations}

\textit{Magnetic field}
\begin{subequations}
\begin{align}
\begin{split}
    \boldsymbol{H}_\text{inc}^j = \sum_{n=1}^{N_\text{max}} \sum_{m=-n}^{n} \bigg\{
    \boldsymbol{\hat r}\frac{1}{r^2} &\bigg[ n(n+1) q_{mn}^j \psi_n(kr) Y_{nm}(\theta,\phi) \bigg] \\
    +\; \boldsymbol{\hat \theta}\frac{k}{r} &\left[ q_{mn}^j \psi_n^\prime(kr) \frac{\partial}{\partial \theta} Y_{nm}(\theta,\phi)
    + m\sqrt{\varepsilon_b} p_{mn}^j \psi_n(kr) \frac{Y_{nm}(\theta,\phi)}{\sin\theta} \right] \\
    +\; \boldsymbol{\hat \phi}\frac{k}{r} &\left[ im q_{mn}^j \psi_n^\prime(kr) \frac{Y_{nm}(\theta,\phi)}{\sin\theta}
    + i\sqrt{\varepsilon_b} p_{mn}^j \psi_n(kr) \frac{\partial}{\partial \theta} Y_{nm}(\theta,\phi) \right] \bigg\}
\end{split}
\end{align}
\begin{align}
\begin{split}
    \boldsymbol{H}_\text{scat}^j = \sum_{n=1}^{N_\text{max}} \sum_{m=-n}^{n} \bigg\{
    \boldsymbol{\hat r}\frac{1}{r^2} &\bigg[ n(n+1) b_{mn}^j \xi_n^{(1)}(kr) Y_{nm}(\theta,\phi) \bigg] \\
    +\; \boldsymbol{\hat \theta}\frac{k}{r} &\left[ n_{mn}^j \xi^{(1)\prime}(kr) \frac{\partial}{\partial \theta} Y_{nm}(\theta,\phi)
    + m\sqrt{\varepsilon_b} a_{mn}^j \xi_n^{(1)}(kr) \frac{Y_{nm}(\theta,\phi)}{\sin\theta} \right] \\
    +\; \boldsymbol{\hat \phi}\frac{k}{r} &\left[ im b_{mn}^j \xi_n^{(1)\prime}(kr) \frac{Y_{nm}(\theta,\phi)}{\sin\theta}
    + i\sqrt{\varepsilon_b} a_{mn}^j \xi_n^{(1)}(kr) \frac{\partial}{\partial \theta} Y_{nm}(\theta,\phi) \right] \bigg\}
\end{split}
\end{align}
\end{subequations}
where $\xi_n^{(1)} = \psi_n - i \chi_n$, $\psi_n$, $\chi_n$ are the Riccati-Bessel function of the first and second kind, and $Y_{nm}$ are the spherical harmonics
\begin{align}
\begin{split}
    \psi_n(x) &= xj_n(x) \\
    \chi_n(x) &= -xy_n(x) \\
    \xi_n^{(1)}(x) &= x[j_n(x) + iy_n(x)] = xh_n^{(1)}(x) \\
    Y_{nm}(\theta, \phi) &= \sqrt{\frac{2n+1}{4\pi}\frac{(n-m)!}{(n+m)!}} P_n^m(\cos \theta) e^{im\phi}
\end{split}
\end{align}

Denoting the coefficients of our convention $\bar a_n$, $\bar b_n$, $\bar p_{mn}$, $\bar q_{mn}$, $\bar a_{mn}$, $\bar b_{mn}$, the two conventions are related by
\begin{align}
\begin{split}
    \bar a_{n} &= - a_{n} \\
    \bar b_{n} &= - b_{n} \\
    \bar p_{mn} &= \frac{k^2}{i^{n-1}}\sqrt{\frac{n(n+1)}{4\pi}} p_{mn} \\
    \bar q_{mn} &= -\frac{k^2}{i^n}\sqrt{\frac{\mu_b}{\varepsilon_b} \frac{n(n+1)}{4\pi}} q_{mn} \\
    \bar a_{mn} &= -\frac{k^2}{i^{n-1}}\sqrt{\frac{n(n+1)}{4\pi}} a_{mn} \\
    \bar b_{mn} &= \frac{k^2}{i^n}\sqrt{\frac{\mu_b}{\varepsilon_b} \frac{n(n+1)}{4\pi}} b_{mn}
\end{split}
\end{align}
Another convention uses a different value for the $E_{mn}$ normalization values \cite{xu1995electromagnetic}
\begin{equation}
    E_{mn} = |E_0|i^n \frac{2n+1}{n(n+1)}
\end{equation}
where $|E_0|$ is the amplitude of the source field.
We have chosen to absorb this amplitude into the $a$, $b$, $p$, and $q$ coefficients.

\section{MiePy: a GMMT Python library}

\subsection{Installation}
\noindent
Install required dependencies:
\begin{itemize}[label={\tiny\raisebox{1ex}{\textbullet}}]
    \item \href{https://cmake.org/install/}{CMake} (C++ build system)
    \item \href{http://eigen.tuxfamily.org/index.php?title=Main_Page}{Eigen} (C++ linear algebra library)
    \item \href{https://www.gnu.org/software/gsl/}{GNU Scientific Library (GSL)}
\item GCC and GFORTRAN
\item Python 3 and pip
\end{itemize}
Run the commands:
\begin{lstlisting}
git clone https://johnapark@bitbucket.org/johnapark/miepy.git --recurse-submodules
cd miepy
pip install . -e --user     # build and install
pytest tests                # run tests
\end{lstlisting}

\subsection{Example script}

Example Python code using MiePy
\begin{lstlisting}
    import miepy
    nm = 1e-9

    # define material and source
    Ag = miepy.materials.Ag()
    source = miepy.sources.plane_wave(polarization=[1,0])

    # build an Ag dimer with radii 50nm separated by 600nm in the x-direction
    dimer = miepy.sphere_cluster(position=[[300*nm,0,0], [-300*nm,0,0]],
                                    radius=50*nm,
                                    material=Ag,
                                    source=source,
                                    wavelength=800*nm,
                                    lmax=2)

    # obtain the cross-sections
    scat, absorb, extinct = dimer.cross_sections()

    # obtain the force and torque on the right particle
    F = dimer.force_on_particle(0)
    T = dimer.torque_on_particle(0)
\end{lstlisting}

\subsection{Target features}

\begin{itemize}
    \item[] \textbf{version 0.3} (released)
        \begin{itemize}[label={\tiny\raisebox{1ex}{\textbullet}}]
            \item 30$\times$ performance for larger clusters
            \item T-matrix formulation for non-spherical particles
            \item Plane-waves and beams can have arbitrary direction and polarization
        \end{itemize}
    \item[] \textbf{version 0.4}
        \begin{itemize}[label={\tiny\raisebox{1ex}{\textbullet}}]
            \item C++ implementation with OpenMP parallelization and Python bindings (using pybind11)
            \item Periodic boundary conditions (1D and 2D)
            \item Symmetry relationships (mirror, discrete rotation)
            \item Additional performance: 
                        (i) A/B remainging symmetries, 
                        (iv) no T-matrix recomputation for identical geometries
                        (v) transition from [2, rmax] $\rightarrow$ [rmax] shape
                    \item Save solution so that it can be reloaded
            \item Any changes to T-matrix/sources
        \end{itemize}
    \item[] \textbf{version 0.5}
        \begin{itemize}[label={\tiny\raisebox{1ex}{\textbullet}}]
            \item Additional functions: 
                       (i) cluster T-matrix,
                       (ii) S-matrix, Mueller/Jones matrix,
                       (iii) spin vs. orbital torque,
                       (iv) local density of states,
                       (v) energy and momentum density
            \item Performance:
                        (i) lmax per particle,
                        (ii) cluster T-matrix re-usage with many sources
                        (iii) computing A/B for all modes using upwards recursion/symmetries
            \item ``unpolarized'' light
            \item Average over orientations calculations
            \item Scene visualization (3D)
            \item Prettier output for main classes using \_\_repr\_\_
            \item Existing TODO items
            \item Improved documentation, examples, tests, and tutorial/introduction for quick start
        \end{itemize}
    \item[] \textbf{future}
        \begin{itemize}[label={\tiny\raisebox{1ex}{\textbullet}}]
            \item Substrates, layered substrates
            \item Beyond the Rayleigh hypothesis for non-spherical particle interactions
            \item More T-matrix options:
                        (i) non-axisymmetric particles, Gaussian spheres/cylinders, layered spheres/spheroids, etc.
                        (ii) chiral materials,
                        (iii) anisotropic materials (expand definition of material class)
                    \item Valid field expansion in circumscribing spheres (possibly using discrete sources)
            \item 3D periodic boundary conditions
            \item Band diagram calculations
            \item Performance:
                      (i) MPI/GPU parallelization
                      (ii) matrix-free solver and rotation-translation-rotation (RTR) coupling method,
                      (iii) approximation for long-range interactions in large systems,
                      (iv) preconditioner for iterative matrix solver based on spatial locality,
                      (v) grid aggregation + FFT to obtain $\mathcal{O}(N \log N$) performance
            \item Time-domain via IFFT
        \end{itemize}
\end{itemize}

\bibliography{generalized_mie_theory}

\end{document}
